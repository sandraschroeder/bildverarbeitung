Beide Algorithmen liefern zufriedenstellende Ergebnisse bei der Verbesserung der Skelettqualität. In beiden
Fällen besteht Pixelkonnektivität und somit ein Zusammenhang zwischen den Skelettkomponenten. Die Ergebnisse
der beiden Algorithmen sind jeweils unterschiedlich aufgrund ihrer Vorgehensweise bei der Traversierung der Punkte. Die Skelette der beiden Algorithmen die geometrischen Eigenschaften des ursprünglichen Objektes gut wieder. Bei der Breitensuche sowie der Tiefensuche werden die Finger der Hand
nachgezeichnet. Nur im Bereich des Handballen entstehen - besonders bei der Breitensuche - Ausläufer im Skelett. Abhängig von der Anwendung muss man entscheiden, ob diese Ausläufer stören könnten.\\
Ein weiterer Faktor, der die Form des modifizierten Skeletts besonders beeinflusst, ist die Suchdistanz zum
Finden der nächsten Nachbarn. Wird sie zu groß gewählt, werden viele Nachbarn gefunden und somit auch
mehrere Wege von einem Punkt aus (Beispiel Breitensuche). Ist sie zu klein, entstehen Lücken im Skelett
und die Konnektivität zwischen den Skelettlinien ist nicht mehr gegeben. Man muss für jedes Bild beziehungsweise für jedes Bildobjekt entscheiden, welche Suchdistanz sich am besten eignet. Dies kann
bei der Anwendung der beiden Algorithmen auf den Videostream der Kinect hinderlich sein, da die optimale Suchdistanz von Bild zu Bild unterschiedlich sein kann.
%TODO: Noch mehr?