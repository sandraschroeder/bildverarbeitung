\chapter{Einleitung}
\Autor{Christopher Kroll}\\ \\
Im Master-Studiengang Informatik an der Universit"at Hamburg ist ein zweisemestriges Projekt vorgesehen. Die Autoren dieser Arbeit belegten im Sommersemester 2012 und Wintersemester 2012/2013 das Projekt 'Bildverarbeitung' unter der Leitung von Prof. Dr. Leonie Dreschler-Fischer.
%TODO: Die anderen beiden auch aufz�hlen?
Der Verlauf und das Ergebnis dieses Projektes wird in dem vorliegenden Dokument dargestellt.
\\ \\
Die Studenten sollten in diesem Projekt die Daten der Microsoft Kinect-Kamera nutzen und diese f"ur ein Thema ihrer Wahl verarbeiten. Nach einer Vorstellung der Kinect und der Programmiersprache Python wurden die Gruppen eingeteilt und das Thema gew"ahlt.  \\\\
Was ist Skelett? Bla bla, warum toll?
Ein Skelett ist ein nutzlicher Deskriptor, um Informationen uber die Region und den Rand eines Objektes kompakt und effizient
zu kodieren. Es gibt die wesentlichen Grundzuge eines Objektes wieder.\\

\section{Aufgabenstellung}
%TODO: 'Spieler' erwaehnen?
Unsere Gruppe entschied sich f"ur die Datenauswertung einer Person ('Spieler'). Dabei war zun"achst das Ziel den Bewegungsablauf bei Sport"ubungen zu analysieren und eine R"uckmeldung zu geben, ob diese richtig ausgef"uhrt wurden. So soll zum Beispiel bei Kniebeugen durch eine Messung des Winkels zwischen Ober- und Unterschenkel der Person eine Hilfestellung gegeben werden. \\ \\ 
Um dieses Ziel zu erreichen muss zun"achst der Spieler von anderen Gegenst"anden im Raum getrennt, also herausgefiltert werden (Segmentierung). F"ur die Bewegungsanalyse ist es hilfreich nicht den gesamten Menschen, wom"oglich noch mit st"orender Kleidung, zu betrachten, sondern nur sein Skelett. Um das Skelett zu erhalten, bot sich die Wahl zwischen schon implementierten Skelettierungsalgorithmen zu benutzen oder dies selbst zu implementieren. Da das Thema des Projektes die Bildverarbeitung und nicht eine Anwendungsprogrammierung ist, fiel die Entscheidung auf die Konzentration auf die Skelettierung. Das Ziel war nun verschiedene Ans"atze zu implementieren und hinsichtlich Qualit"at und Leistung zu vergleichen.  \\ \\
%anforderungen
TODO skelettdefinition
TODO anforderungen
\section{Aufbau der Projektarbeit}