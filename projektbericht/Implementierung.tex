\chapter{Implementierung der Algorithmen}
\Autor{Sandra Schröder}\\\\
Im Rahmen des Projekts wurden zwei Algorithmen für die Skelettierung implementiert. Dies ist zum einen
die Skelettierung nach Thinning nach dem Algorithmus der in Abschnitt \ref{subsec:fastparallel} beschrieben wurde. Die Skelettierung mittels Distanztransformation wurde nach einer eigenen Idee entwickelt und
umgesetzt.\\
Die Skelettierung läuft in zwei Schritten ab. Erst wird der Spieler segmentiert. Man erhält als
Ergebnis ein Binärbild, welches im zweiten Schritt weiterverarbeitet wird, um ein Skelett zu extrahieren. 
Die Segmentierung des Spielers ist bei beiden Ansätzen identisch.
In diesem Kapitel wird die Umsetzung der Algorithmen anhand signifikanten Codeausschnitten der Implementierungen vorgestellt. Ein Überblick über die technische Umsetzung gibt einen Eindruck über die
verwendeten Programmiersprachen und Arbeitsumgebungen.
\section{Technische Umsetzung}
\Autor{Christopher Kroll}
Die meisten Programme wurden mit der Programmiersprache \emph{Python} umgesetzt. Aus Performanzgründen 
erfolgten einige Umsetzungen in \emph{C++}. -> oder in arbeitsumgebung rein?\\
\section{Spieler-Segmentierung}
\Autor{Sandra Schröder}\\\\
Algorithmus oder nur beschreiben??\\
Es wird nur anhand der Tiefe segmentiert. Um nicht für jeden einzelnen Pixel die Bedingung zu überprüfen, ob er den Schwellwert für die Segmentierung überschreitet beziehungsweise unterschreitet, wird die effiziente Numpy-Funktion \texttt{logical\_and} benutzt, die global auf dem Bild arbeitet und für das gesamte Bild die Schwellwertbedingung prüft. Für den Schwellwert werden zwei 
Werte definiert, um ein Intervall festzulegen, in dem sich das Objekt befinden darf. 
\section{Thinning}  % Würde ich mit anfangen da ich diese Woche fast keine Zeit mehr hab was zu machen (Johannes)
\section{Skelettierung aus der Distanztransformation}
%Sandra
\Autor{Sandra Schröder}\\\\
Die Skelettierung anhand der Distanztransformation läuft folgendermaßen ab:
\begin{itemize}
\item Bestimmen der Distanztransformation des Binärbildes
\item Berechne den Gradientenbetrag auf der Distance Map
\item Differenz zwischen dem Gradientenbild und der Distance Map bilden
\end{itemize}