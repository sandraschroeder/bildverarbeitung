\chapter{Implementierung der Algorithmen}
\Autor{Sandra Schröder}\\\\
Im Rahmen des Projekts wurden zwei Algorithmen für die Skelettierung implementiert. Dies ist zum einen
die Skelettierung nach Thinning nach dem Algorithmus der in Abschnitt \ref{subsec:fastparallel} beschrieben wurde. Die Skelettierung mittels Distanztransformation wurde nach einer eigenen Idee entwickelt und
umgesetzt.\\
Die Skelettierung läuft in zwei Schritten ab. Erst wird der Spieler segmentiert. Man erhält als
Ergebnis ein Binärbild, welches im zweiten Schritt weiterverarbeitet wird, um ein Skelett zu extrahieren. 
Die Segmentierung des Spielers ist bei beiden Ansätzen identisch.
In diesem Kapitel wird die Umsetzung der Algorithmen anhand signifikanten Codeausschnitten der Implementierungen vorgestellt. Ein Überblick über die technische Umsetzung gibt einen Eindruck über die
verwendeten Programmiersprachen und Arbeitsumgebungen.
\section{Technische Umsetzung}
\Autor{Christopher Kroll}\\
F"ur die Implementierung wurden die bereitgestellten iMacs am Informatikum in Stellingen gew"ahlt. Der Vorteil war neben der Performanz die schon eingerichtete Arbeitsumgebung.\\
Um die Verbindung zu der Kinect herzustellen, wurde das quelloffene Framework libfreenect  der OpenKinect-Gruppe benutzt. Libfreenect bietet Treiber und Bibliotheken, um zum Beispiel die Bilder der Kamera anzuzeigen oder den eingebauten Motor zu steuern. \\
Um die empfangenen Bilder verarbeiteten zu k"onnen wurde die Bibliothek OpenCV eingesetzt. Sie unterst"utzt unter anderem beim Speichern und Laden von Bildern und bei der Segmentierung. \\
Der Gro"steil der Programmiierung erfolgte in der Sprache Python. Hierbei handelt es sich um eine leicht zu erlernende Skriptsprache. Allerdings traten im Laufe des Projektes Performanzprobleme bei der Implementierung des Thinning-Algorithmus auf. Aus diesem Grund wurde "uber einen Python-Wrapper der in C++ implementierte Algorithmus eingebaut. N"aheres dazu wird in Kapitel \ref{implThinning}
Als Entwicklungsumgebung wurde spyder f"ur die Python-Programmierung und Xcode f"ur die C++-Programmierung benutzt.
TODO numpy
\section{Spieler-Segmentierung}
\Autor{Sandra Schröder}\\\\
Es wird anhand der Tiefeninformation segmentiert, die die Kinect liefert. Der Open-Source-Treiber für die Kinect - \emph{Freenect} - bietet Funktionen für den Zugriff auf die
Tiefenwerte. Die Funktion \texttt{pretty\_depth} des Moduls
\texttt{frame\_convert} normiert die Tiefenwerte auf das 
Intervall $[0,...,255]$. 
\lstset{
caption={Tiefenwerte zurückgeben}
\label{lst:getdepth}
}
\lstinputlisting{./listing/getdepth.py}
Um nicht für jeden einzelnen Pixel die Bedingung zu überprüfen, ob er den Schwellwert für die Segmentierung überschreitet beziehungsweise unterschreitet, wird die effiziente Numpy-Funktion \texttt{logical\_and} benutzt, die global auf dem Bild arbeitet und für das gesamte Bild die Schwellwertbedingung prüft. Eine pixelgenaue Überprüfung wäre mit Python eine ineffiziente Lösung.\\ Für den Schwellwert werden zwei 
Werte definiert, um ein Intervall festzulegen, in dem sich das Objekt befinden darf. In Listing \ref{lst:spielersegmentierung} legen die Variablen \texttt{current\_depth} und \texttt{threshold} das Intervall fest. 
\lstset{
caption={Spielersegmentierung in Python}
\label{lst:spielersegmentierung}
}
\lstinputlisting{./listing/player_segmentation.py}
Da die Funktion auf Numpy-Arrays arbeitet, muss das Bildobjekt zuvor in ein Array umgewandelt werden. 
Es wurden vorgefertigte Funktionen von OpenCV benutzt, die diese Konvertierung vornehmen. 
\section{Skelettierung mittels Thinning} 
\label{implThinning}
\Autor{Christopher Kroll}

\section{Skelettierung mittels Distanztransformation}
%Sandra
\Autor{Sandra Schröder}\\\\
Der theoretische Ablauf der Skelettierung wurde bereits im Kapitel \ref{ch:Skelettierung} beschrieben. Zur
Rekapitulation werden die Schritte kurz aufgezählt. \\
Die Skelettierung anhand der Distanztransformation läuft folgendermaßen ab:
\begin{itemize}
\item Bestimmen der Distanztransformation des Binärbildes
\item Berechne den Gradientenbetrag auf der Distance Map
\item Differenz zwischen dem Gradientenbild und der Distance Map bilden
\end{itemize}
Die Distance Map kann mit einer Funktion aus der Bildverarbeitungsbibliothek \emph{OpenCV} einfach berechnet
werden. Die Funktion (Listing \ref{lst:disttransform}) erwartet als Eingabe das Originalbild (\texttt{img}) und ein Bild (\texttt{dist\_img}), um
das Ergebnis speichern zu können (gleiche Größe und Dimension wie das Originalbild). Eine weitere Möglichkeit, die die Funktion bietet, ist die Angabe einer Metrik, nach der der Abstand eines Pixels zum
Hintergrund bestimmt wird. Es wurde die euklidische Metrik benutzt.
\lstset{
caption={Berechnen der Distance Map des Spielers.}
\label{lst:disttransform}
}
\lstinputlisting{./listing/distancetransform.py}
Zur Bestimmung des Gradientenbetrages der Distance Map wurde die Numpy-Funktion \texttt{gaussian\_gradient\_magnitude} genutzt. Die Funktion berechnet den Gradientenbetrag mit Ableitungen
der Gaussfunktion. Die Variable \texttt{sigma} ist dementsprechend für die Standardabweichung $\sigma$ 
des Gaussfilters gedacht \footnote{Für weitere Dokumentationen siehe: http://docs.scipy.org/doc/scipy/reference/generated/scipy.ndimage.filters.gaussian\_gradient\_magnitude.html}. Das Ergebnis dieser Funktion wird in ein Bildobjekt
konvertiert und entsprechend festgelegter Schwellwerte (\texttt{lowerbound}, \texttt{upperbound}) segmentiert.
\lstset{
caption={Gradientenbetrag der Distance Map und Segmentierung des Gradientenbetragsbildes.}
\label{lst:gradient}
}
\lstinputlisting{./listing/gradient.py}
Zur Differenzbildung und endgültigen Berechnung des Distanzskelettes werden die Bildobjekte in Arrays umgewandelt. 
Diese Arrays können einfach voneinander abgezogen werden. 
\lstset{
caption={Differenz zwischen Distance Map und segmentiertem Gradientenbetrag}
\label{lst:difference}
}
\lstinputlisting{./listing/difference.py}
Bei der Implementierung wurden in keinem Fall Operationen ausgeführt, die auf einzelne Pixel zugreifen. Situationen, in denen
pixelgenaue Operationen durchgeführt werden könnten, wurden
umgangen, indem Arrayoperationen oder Funktionen aus der
OpenCV-Bibliothek benutzt wurden. Ein pixelgenauer Zugriff
könnte bei einer Interpretersprache wie Python zu einer sehr langsamen
Ausführung der Skelettberechnung führen.