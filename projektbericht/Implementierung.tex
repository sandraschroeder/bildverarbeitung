\chapter{Implementierung der Algorithmen}
\Autor{Sandra Schröder}\\\\
Dieses Kapitel beschreibt die Algorithmen für Thinning und die Distanztransformation.\\
Der Algorithmus erhält einen Videostream von Tiefenbilder der Kinect. Diese werden für die
Skelettierung weiterverarbeitet. \\
Bevor Skelettierungsalgorithmen angewendet werden können, muss ein Binärbild des Bildes erzeugt werden, 
das die Kinect liefert. Dafür wurde eine Segmentierung implementiert, die Objekte, die sich vor der
Kinect befinden, weiß markiert. Für die Segmentierung wurde nur die Tiefeninformation der Kinect genutzt.
Anschließend werden die Algorithmen angewandt. 
\begin{algorithm}
\begin{algorithmic}[1]
\Procedure{DoSkeletonization}{Image}
\While{True}
\State{Segmentiere Spieler}
\State{Wende Thinning oder Distanztransformation an}
\EndWhile
\EndProcedure
\end{algorithmic}
\end{algorithm}
\section{Wahl der Algorithmen}
%TODO
\section{Die Arbeitsumgebung oder: Technische Umsetzung??}
%TODO
Die meisten Programme wurden mit der Programmiersprache \emph{Python} umgesetzt. Aus Performanzgründen 
erfolgten einige Umsetzungen in \emph{C++}. -> oder in arbeitsumgebung rein?\\
\section{Spieler-Segmentierung}
\Autor{Sandra Schröder}\\\\
Algorithmus oder nur beschreiben??\\
Es wird nur anhand der Tiefe segmentiert. Um nicht für jeden einzelnen Pixel die Bedingung zu überprüfen, ob er den Schwellwert für die Segmentierung überschreitet beziehungsweise unterschreitet, wird die effiziente Numpy-Funktion \texttt{logical\_and} benutzt, die global auf dem Bild arbeitet und für das gesamte Bild die Schwellwertbedingung prüft. Für den Schwellwert werden zwei 
Werte definiert, um ein Intervall festzulegen, in dem sich das Objekt befinden darf. 
\section{Thinning}  % Würde ich mit anfangen da ich diese Woche fast keine Zeit mehr hab was zu machen (Johannes)
\section{Skelettierung aus der Distanztransformation}
%Sandra
\Autor{Sandra Schröder}\\\\
Die Skelettierung anhand der Distanztransformation läuft folgendermaßen ab:
\begin{itemize}
\item Bestimmen der Distanztransformation des Binärbildes
\item Berechne den Gradientenbetrag auf der Distance Map
\item Differenz zwischen dem Gradientenbild und der Distance Map bilden
\end{itemize}