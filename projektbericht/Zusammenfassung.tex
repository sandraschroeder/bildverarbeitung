\chapter{Zusammenfassung und Ausblick}
\label{ch:ausblick}
\Autor{Christopher Kroll}
\section{Zusammenfassung}
\section{Ausblick}
Vorschlag: Der Vergleich der Algorithmen hat gezeigt, dass sich abhängig vom Algorithmus die Skelettqualität unterscheidet. Auch die 
Laufzeit ist ein wichtiges Unterscheidungsmerkmal. Vor allem im Hinblick auf die Anforderung Echtzeitfähigkeit.\\
Die Skelette, die mit den beiden Algorithmen bestimmt wurden, eignen sich beiden für weitere Anwendungen. Allerdings muss bei beiden Algorithmen optimiert werden. Beim Thinning-Algorithmus ist dies die Laufzeit, bei der Skelettierung mittels Distanztransformation ist dies die Skelettqualität. Verbesserung des Distanzskeletts sind vielversprechend. Der nächste Schritt wäre die Integration in die Kinect-Umgebung. Es muss
getestet werden, wie performant das ist...\\
\begin{itemize}
\item Der Vergleich hat gezeigt, dass Algorithmen unterschiedliche Ergebnisse liefern
\item in qualität und laufzeit
\item weiternutzen der skelette
\item beide eignen sich
\item vor allem: anwenden der verbesserten version des distanzskelettes: performant mit der kinect oder nicht?
\item wenn nicht: kann man das verbessern?
\item grundstein dafür wurde gelegt. 
\end{itemize}
