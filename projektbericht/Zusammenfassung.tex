\chapter{Zusammenfassung und Ausblick}
\label{ch:ausblick}
\Autor{Christopher Kroll}
\section{Zusammenfassung}
F"ur alle Teammitglieder war das Arbeiten mit Python und der Kinect Neuland. In der Anfangsphase des Projektes musste sich zun"achst in die Programmiersprache und die Funktionsweise der Kamera eingearbeitet werden. Welche Daten die Kinect auf welche Weise liefert und wie diese mit den Frameworks aufgenommen und benutzt werden k"onnen, musste untersucht werden.\\
Damit nicht das komplette Blickfeld der Kamera, sondern nur das relevante Objekt, also der Mensch vor der Kamera, skelettiert wird, musste dieser segmentiert werden.\\Aufbauend auf diesem Ergebnis konnte sich mit der Skelettierung und den unterschiedlichen Verfahren gek"ummert werden. In dieser Phase wurde das Ziel der Projektarbeit angepasst. Wurde zun"achst angepeilt ein Bewegungsanalyseprogramm zu schreiben, wurde nun die Konzentration auf eben diese Verfahren gelegt und das Vorhaben war es jetzt Skelettierungsalgorithmen zu implementieren und zu vergleichen.\\Die Entscheidung fiel auf eine Distanztransformation und ein Thinning-Verfahren. W"ahrend bei der Distanztransformation ein eigener Ansatz gew"ahlt wurde, entschied man sich durch die Papervorstellung im Seminar einen bereits etablierten Algorithmus f"r das Thinning zu w"ahlen. Trotz einigen Schwierigkeiten, wie dem Laufzeitproblem in Python und dem Wechsel des Algorithmus, konnten am Ende gute Ergebnisse erzielt und die beiden Verfahren miteinander verglichen werden. \\
Es stellte sich heraus, dass das Thinning zwar ein gutes Skelett liefert, aber durch den iterativen Prozess zu langsam ist. Der Wechsel der Programmiersprache brachte zwar einen Leistungszuwachs, dieser war jedoch noch immer zu unbefriedigend bei Anwendung auf Bewegtbilder. F"ur statische Bilder ist dieser Ansatz jedoch geeignet. \\ Die Distanztransformation ist dagegen echtzeitf"ahig. Allerdings liefert dieser Ansatz nicht so ein gutes Skelett wie das Thinning. So ist die Pixelkonnektivit"at nicht gegeben. 
\section{Ausblick}
Der Vergleich der Algorithmen hat gezeigt, dass sich abhängig vom Algorithmus die Skelettqualität unterscheidet. Auch die 
Laufzeit ist ein wichtiges Unterscheidungsmerkmal. Vor allem im Hinblick auf die Anforderung Echtzeitfähigkeit.\\
Die Skelette, die mit den beiden Algorithmen bestimmt wurden, eignen sich beiden für weitere Anwendungen. Allerdings muss bei beiden Algorithmen optimiert werden. Beim Thinning-Algorithmus ist dies die Laufzeit, bei der Skelettierung mittels Distanztransformation ist dies die Skelettqualität. Verbesserung des Distanzskeletts sind vielversprechend. Der nächste Schritt wäre die Integration in die Kinect-Umgebung. Es muss
getestet werden, wie performant dies ist. \\ Ist die Skelettierung optimiert, kann man das Ergebnis nutzen, um Anwendungen, wie zum Beispiel die erw"ahnte und zun"achst vorgesehene Bewegungsanalyse zu erstellen.