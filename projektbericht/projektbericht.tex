\documentclass[appendixprefix,a4paper,bibliography=totoc,twoside=true,11pt,DIV=11,BCOR=6mm,headsepline,pointlessnumbers]{scrbook}
\usepackage[ngerman]{babel}

\usepackage[T1]{fontenc}
\usepackage[utf8]{inputenc}
%\usepackage{amsmath, amsthm, amssymb} %Formatierung der Formeln von MPues 
\usepackage{amsmath,amssymb,amsthm,upgreek,bbm,textcomp,mhchem,siunitx}
\usepackage{mathtools}
\usepackage{bibgerm}
\usepackage{graphicx} 
%\graphicspath{{./src/}} 		
\usepackage[marginal, hang]{footmisc}        %F�r andere Fu�noten. Doku: 
%http://www.cs.brown.edu/system/software/latex/doc/footmisc.pdf, MUSS VOR setspace geladen werden
\usepackage{setspace}                        %f�r 1,5er Zeilenabstand
%\usepackage{longtable}
%\usepackage[automark]{scrpage2}
%\usepackage{lineno}                        %Zeilennummerierung
\usepackage{esvect}                            %Vektorpfeile: Benutzung mit \vv{abc}
%\usepackage{subfig}                         %Mehrere Bilder
%\SI{6,1}{\giga\hertz}.
\usepackage{lmodern}                        %latin modern schriftarten laden
\usepackage{color}                            %Farbiger Text in Tex. Benutzen mit \textcolor[red}{hier der text}
%\usepackage{booktabs}                        %F�r Excel2Latex
\usepackage{multirow}
%\usepackage{bigfoot}
\usepackage[labelfont=bf, format=plain, font=small]{caption}
\usepackage[rightcaption]{sidecap}
\usepackage[hyperfootnotes=false]{hyperref}                        
%links in pdf erzeugen, als letztes Paket setzten sonst gibts mit pagenumbering stress
\usepackage[babel,german=quotes]{csquotes}        % F�r einheitliche Anf�hrungszeichen
\usepackage{placeins} %F�r Float-Barrier und Kontrolle �ber Floating-Objekte
%\usepackage{algorithm2e}
%\usepackage{algpseudocode}

%\usepackage[firstpage]{draftwatermark}
%\usepackage{draftwatermark}

\usepackage{algpseudocode}
\usepackage[Algorithmus]{algorithm}
% \usepackage[Algorithmus]{algorithm}
% \usepackage{algorithmicx}
% \usepackage{algcompatible}
% \usepackage{algorithm2e}
\usepackage{ragged2e}
\renewcommand{\sffamily}{\fontfamily{phv}\selectfont}          %�berschriften in Helvetica

\begin{document}
%%%%%%%%%%%%%%%%%%%%%%%%%%%%%%%%%%%%%%%%%%%%%%%%%%%%%%%%%%%%
%Schalter Titelseite
%
% \titlehead{titlehead}%
% \subject{}
% \subtitle{}
%
\title{Implementierung von Skelett-Algorithmen mit dem Kinect-Sensor}
\author{Arbeitsbereich Kognitive Systeme (KOGS)\\ Fachbereich Informatik\\ Universität Hamburg}
\publishers{Projektbarbeit\\ Projekt Bildverarbeitung \\Johannes Böhler, Christopher Kroll, Sandra Schröder\\  \vspace{2 cm} Sommersemester 2012 bis Wintersemester 2012/2013}
%\uppertitleback{uppertitleback}
\lowertitleback{}
\date{}
%%%%%%%%%%%%%%%%%%%%%%%%%%%%%%%%%%%%%%%%%%%%%%%%%%%%%%%%%%%%%
%
% 
% \begin{titlepage}
% \thispagestyle{empty}
% \begin{center}
% 
% {\LARGE Bachelorarbeit: \\ [1cm]
% Ballerkennung mit Formwahrscheinlichkeitsverteilungen im Robocup\\[2cm] }
% 
% {\Large von
% \\ [0.5cm]  Sandra Schröder\\
% (6060939)\\
% geboren am \\ 12.05.1989\\ [2cm]}
% 
% {\Large im Studiengang
% Informatik\\
% Arbeitsbereich Technische Informatiksysteme \\
% Universit\"at Hamburg\\[3cm]}
% 
% 
% \end{center}
% Erstgutachter: Prof. Dr.-Ing. Dietmar P. F. Möller\\
% Zweitgutachter: Prof. Dr. rer. nat. Leonie Dreschler-Fischer
% \end{titlepage}
\maketitle
\chapter{Einleitung}
\section{Aufbau der Projektarbeit}
%TODO: Muss das wirklich sein?
\chapter{Aufgabenstellung}
%Sandra
\section{Verwendung des Kinect-Sensors}
\section{Anforderungen}

\chapter{Die Skelettierung}
\section{Thinning}
\section{Distanztransformation}
%Sandra: Theorie über Skelettierung mit der Distanztransformation
\section{Weitere Verfahren}
\section{Verwandte Arbeiten}
\subsection{Extracting Skeletons From Distance Maps}
%Sandra
\subsection{A Fast Parallel Algorithm for Thinning Digital Patterns} 


\chapter{Implementierung der Algorithmen}
\section{Wahl der Algorithmen}
\section{Die Arbeitsumgebung}
\section{Spieler-Segmentierung}
%TODO: Tiefeninformation, Verbesserung des Tiefenbilds, Nachbesserung des segmentierten Bildes -> Dilatation
\section{Thinning}
\section{Distanztransformation}
%Sandra
\chapter{Ergebnisse}
%Sandra
%TODO: Kriterien für den Vergleich der Algorithmen -> Eigenschaften eines Skeletts, Laufzeit
%TODO: Anwendung programmieren?
\begin{itemize}
	\item Vergleich der Algorithmen
	\item Anhand der Kriterien
	\begin{itemize}
		\item Erhaltung der Topologie
		\item Pixelkonnektivität
		\item Zentriert
		\item 1 Pixel breit
	\end{itemize}
	\item Echtzeitfähigkeit -> Messungen machen -> Vergleich
	\item Verbesserung des Skeletts (Distanztransformation) mit Breitensuche um Pixelkonnektivität zu erreichen -> Weitere Verbesserungen? -> Ohne Features sondern anhand der weißen Pixel
	\item Anwendung: Vergleich von Posen -> Features bestimmen. Vllt sowas wie "Spannweite" der Pose in x und in y Richtung (Abstand des "linkesten" zum "rechtesten" Pixel). 
\end{itemize}
\chapter{Zusammenfassung}
\chapter{Fazit und Ausblick}
\section{Fazit}
\section{Ausblick}
\chapter{TODO Anhang}
Tabelle: Wer hat was geschrieben?\\
Vollständiger Quellcode
\end{document}